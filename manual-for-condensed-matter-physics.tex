% !TEX TS-program = pdflatex
% !TEX encoding = UTF-8 Unicode

% This is a simple template for a LaTeX document using the "article" class.
% See "book", "report", "letter" for other types of document.

\documentclass[11pt]{book} % use larger type; default would be 10pt

\usepackage[utf8]{inputenc} % set input encoding (not needed with XeLaTeX)

%%% Examples of Article customizations
% These packages are optional, depending whether you want the features they provide.
% See the LaTeX Companion or other references for full information.

%%% PAGE DIMENSIONS
\usepackage[top=2.5cm, bottom=2.5cm, outer=2.5cm, inner=4cm]{geometry} % to change the page dimensions
\geometry{letterpaper} % or letterpaper (US) or a5paper or....
%\geometry{margin=2.5cm} % for example, change the margins to 2 inches all round
% \geometry{landscape} % set up the page for landscape
%   read geometry.pdf for detailed page layout information

\usepackage{graphicx} % support the \includegraphics command and options

% \usepackage[parfill]{parskip} % Activate to begin paragraphs with an empty line rather than an indent

%%% PACKAGES
\usepackage{booktabs} % for much better looking tables
\usepackage{array} % for better arrays (eg matrices) in maths
\usepackage{paralist} % very flexible & customisable lists (eg. enumerate/itemize, etc.)
\usepackage{verbatim} % adds environment for commenting out blocks of text & for better verbatim
%\usepackage{subfig} % make it possible to include more than one captioned figure/table in a single float
% These packages are all incorporated in the memoir class to one degree or another...

%%% HEADERS & FOOTERS
\usepackage{fancyhdr} % This should be set AFTER setting up the page geometry
\pagestyle{fancy} % options: empty , plain , fancy
\renewcommand{\headrulewidth}{0pt} % customise the layout...
%\lhead{}\chead{}\rhead{}
%\lfoot{}\cfoot{\thepage}\rfoot{}

%%% SECTION TITLE APPEARANCE
\usepackage{sectsty}
\allsectionsfont{\sffamily\mdseries\upshape} % (See the fntguide.pdf for font help)
% (This matches ConTeXt defaults)

%%% ToC (table of contents) APPEARANCE
\usepackage[nottoc,notlof,notlot]{tocbibind} % Put the bibliography in the ToC
\usepackage[titles,subfigure]{tocloft} % Alter the style of the Table of Contents
\renewcommand{\cftsecfont}{\rmfamily\mdseries\upshape}
\renewcommand{\cftsecpagefont}{\rmfamily\mdseries\upshape} % No bold!

%%% ADDED BY WENJIE CHEN
\usepackage{wrapfig}
\usepackage{subfigure}
%\usepackage{xcolor}
\usepackage[svgnames]{xcolor}
\usepackage{bm}
\usepackage{amsmath}
\usepackage{amssymb}
\usepackage{booktabs}
\usepackage{multirow}
\usepackage{extarrows}
\usepackage{hyperref}
\newcommand{\angstrom}{\textup{\AA}}

%%% END Article customizations

%%% The "real" document content comes below...

%\title{\textbf{Manual of Condensed Matter Physics}}
%\author{Wenjie Chen}
%\date{} % Activate to display a given date or no date (if empty),
         % otherwise the current date is printed 

\begin{document}
%\maketitle

\begin{titlepage} % Suppresses headers and footers on the title page

	\raggedleft % Right align everything
	
	\vspace*{\baselineskip} % Whitespace at the top of the page
	
	%------------------------------------------------
	%	Author
	%------------------------------------------------
	
	{\Large Wenjie Chen} % Author name
	
	\vspace*{0.167\textheight} % Whitespace before the title
	
	%------------------------------------------------
	%	Title and subtitle
	%------------------------------------------------
	
	\textbf{\LARGE A Manual for}\\[\baselineskip] % First title line
	
	\textbf{\textcolor{Blue}{\Huge Condensed Matter Physics}}\\[\baselineskip] % Main title line which draws the focus of the reader
	
	{\Large \textit{for beginners}} % Subtitle
	
	\vfill % Whitespace between the titles and the publisher
	
	%------------------------------------------------
	%	Publisher
	%------------------------------------------------
	
	{\large International Center for Quantum Materials\\Peking University} % Publisher and logo
	
	\vspace*{2\baselineskip} % Whitespace at the bottom of the page

\end{titlepage}

\tableofcontents

\chapter{Magnetism}

\begin{center}
\fbox{
\begin{minipage}[c]{0.8\textwidth}
\vspace{0.5cm}
\begin{center}
Science is rooted in conversations.
\end{center}
\begin{flushright}
------ Werner Heisenberg (1901 - 1976)
\end{flushright}
\vspace{0cm}
\end{minipage}
}
\end{center}

\section{Magnetic Properties}
\subsection{Magnetic susceptibility}
\subsubsection{Definition}
\textbf{Magnetic susceptibility} (denoted $\chi$) is a dimensionless proportionality constant that indicates the degree of magnetization of a material in response to an applied magnetic field. It indicates whether a material is attracted into or repelled out of a magnetic field. Quantitative measures of the magnetic susceptibility also provide insights into the structure of materials, providing insight into bonding and energy levels.\footnote{See wikipedia page: \href{https://en.wikipedia.org/wiki/Magnetic_susceptibility}{Magnetic susceptibility}.}

The definition of magnetic susceptibility (volume susceptibility) is as followed
\begin{equation}
\mathbf{M} = \chi \mathbf{H}
\end{equation}
where $\mathbf{M}$ is the magnetization of the material (the magnetic dipole moment per unit volume), and $\mathbf{H}$ is the magnetic field strength.

A material can be \textbf{paramagnetic} ($\chi > 0$) or \textbf{diamagnetic} ($\chi < 0 $) depending on whether the magnetic field in it is strengthened or weakened by the induced magnetization.

While volume susceptibility is a dimensionless constant, mass susceptibility and molar susceptibility are \textbf{not}. They are defined as
\begin{equation}
\chi_{\rm mass} = \frac{\chi}{\rho}
\end{equation}
and
\begin{equation}
\chi_{\rm mol} = M \chi_{\rm mass} = \frac{M\chi}{\rho}
\end{equation}
where $\rho$ is the density in $\rm kg/m^3$ and $M$ is molar mass in $\rm kg/mol$.

\subsubsection{Different Unit Systems}
Noted that in SI units, the magnetic induction $\mathbf{B}$ is related to $\mathbf{H}$ by the relationship
\begin{equation}
\mathbf{B} = \mu_0 (\mathbf{H} + \mathbf{M}) = \mu_0(1+\chi)\mathbf{H} = \mu \mathbf{H}
\end{equation}
where $\mu_0$ is the vacuum permeability, and $(1+\chi)$ is the relative permeability of the material. Thus the volume magnetic susceptibility $\chi$ and the magnetic permeability $\mu$ are related by the following formula
\begin{equation}
\mu =\mu_{0}\left(1+\chi\right).
\end{equation}

However in Gaussian units (or cgs emu, which is the same for magnetic properties), the magnetic induction $\mathbf{B}$ is related to $\mathbf{H}$ by the relationship
\begin{equation}
\mathbf{B} = \mathbf{H} + 4\pi \mathbf{M} = (1+4\pi\chi)\mathbf{H}.
\end{equation}

For a conversion between SI units and Gaussian units, please refer to \href{https://www.qdusa.com/sitedocs/UnitsChart.pdf}{this table}.

\subsubsection{Susceptibility Tensor}
The magnetic susceptibility of most crystals is not a scalar quantity. Magnetic response $\mathbf{M}$ is dependent upon the orientation of the sample and can occur in directions other than that of the applied field $\mathbf{H}$. In these cases, volume susceptibility is defined as a tensor
\begin{equation}
{M}_{i} = {H}_{j} \chi_{ij}
\end{equation}
where $i$ and $j$ refer to the directions (e.g., $x$ and $y$ in Cartesian coordinates) of the applied field and magnetization, respectively.

\subsubsection{Differential Susceptibility}
In ferromagnetic crystals, the relationship between $\mathbf{M}$ and $\mathbf{H}$ is not linear. To accommodate this, a more general definition of differential susceptibility is used
\begin{equation}
\chi_{ij}=\frac{\partial M_{i}}{\partial H_{j}}
\end{equation}
where $\chi_{ij}$ is a tensor derived from partial derivatives of components of $\mathbf{M}$ with respect to components of $\mathbf{H}$. 

\subsection{Ne\'{e}l Temperature}

\subsection{Curie Temperature}

\section{Ferromagnetism}

\section{Antiferromagnetism}

\section{Ising Model}

More text.

\chapter{Miscellany}
\begin{center}
Science is rooted in conversations.
\end{center}

\begin{flushright}
------ Werner Heisenberg (1901 - 1976)
\end{flushright}

\end{document}
